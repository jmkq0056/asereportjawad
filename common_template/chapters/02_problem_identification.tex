\chapter{Recognizing Development Challenges}
\label{ch:problem-identification}

Before Djøf Trade Union adopted Scrum, their development team faced significant challenges that hindered their ability to deliver value efficiently. Understanding these problems is essential to appreciate why they chose Agile practices and how those practices addressed their specific needs. The Scrum framework emphasizes that transparency enables inspection and adaptation \textit{\parencite{schwaber2020scrum}}, but achieving transparency first requires identifying what is unclear. Empirical evidence demonstrates that teams struggle most when organizational blockers around stakeholder alignment, feedback loops, and continuous learning remain unaddressed \textit{\parencite{verwijs2021scrum}}.

\section{Key Challenges}
\label{sec:key-challenges}

Djøf Trade Union faced a fundamental problem with how work arrived and was prioritized. Tasks would emerge ad-hoc without clear consideration for team capacity or strategic importance, creating an environment where developers constantly reacted to incoming requests rather than working proactively toward meaningful goals.

The situation became particularly problematic when dealing with legacy systems. Ahmed described a project where he was asked to add a simple checkbox to the member registration system for union unemployment insurance sign-up \textbf{(Transcript: 00:02:23--00:02:34)}. The legacy membership system had security restrictions that prevented making changes without being logged in as an actual member, and creating test members was not permitted for data protection reasons \textbf{(Transcript: 00:03:04--00:03:33)}. He spent two weeks to a month working on this task, only to have it completely abandoned \textbf{(Transcript: 00:04:32--00:04:48)}.

The task should never have reached Ahmed in the first place. It needed approval from the legal department regarding GDPR compliance and coordination with the third-party vendor who controlled the membership system's security settings \textbf{(Transcript: 00:04:48--00:05:17)}. There was no gate-keeping process to ensure that tasks were actually achievable within existing constraints before developers invested time in them. This pattern repeated itself across various projects. Developers would receive assignments that looked simple on the surface but contained hidden dependencies, unclear stakeholder expectations, or technical constraints that had not been discovered.

% Figure: Problem Workflow - Before Scrum Adoption
% Simple narrative flow showing how work arrived chaotically

\begin{figure}[htbp]
    \centering
    \begin{tikzpicture}[
        box/.style={rectangle, draw, fill=blue!10, text width=2.8cm, align=center, minimum height=1cm, font=\small},
        problem/.style={rectangle, draw, fill=red!20, text width=2.8cm, align=center, minimum height=1cm, font=\small},
        arrow/.style={->, >=stealth, thick}
    ]

    % Top section - Business requests
    \node[box] (business1) at (0,4) {Business Request 1\\(GDPR unclear)};
    \node[box] (business2) at (3.5,4) {Business Request 2\\(No priority)};
    \node[box] (business3) at (7,4) {Business Request 3\\(Legacy constraints)};

    % Middle - Direct to developers (the problem)
    \node[problem] (devs) at (3.5,2) {Development Team\\(Overwhelmed)};

    % Arrows showing chaotic flow
    \draw[arrow, red] (business1) -- (devs);
    \draw[arrow, red] (business2) -- (devs);
    \draw[arrow, red] (business3) -- (devs);

    % Bottom - Outcomes
    \node[problem] (waste) at (1,0) {Wasted Effort\\(2-4 weeks lost)};
    \node[problem] (unclear) at (6,0) {Unclear Requirements\\(Late discovery)};

    \draw[arrow, red, dashed] (devs) -- (waste);
    \draw[arrow, red, dashed] (devs) -- (unclear);

    % Labels
    \node[above=0.1cm of business2, font=\small\bfseries] {Ad-hoc Work Arrival};
    \node[left=0.3cm of devs, font=\scriptsize, text=red, text width=1.8cm, align=right] {No filtering\\No priority\\No refinement};

    \end{tikzpicture}
    \caption{Chaotic work arrival before Scrum adoption. Business requests flowed directly to developers without proper refinement, prioritization, or feasibility analysis, leading to wasted effort and late discovery of blockers.}
    \label{fig:jawad-problem-workflow}
\end{figure}


\section{Problem Analysis}
\label{sec:problem-analysis}

The core issue was the absence of structured processes to connect business needs with development capacity. When the business side wanted something built, there was no intermediary to translate those desires into well-defined, technically feasible requirements. Business stakeholders assumed their requests were feasible. Developers accepted tasks without fully understanding the business context or questioning whether all necessary approvals were in place. Both sides operated with incomplete information.

Without a clear backlog or product owner role, everything seemed equally urgent. There was no mechanism to evaluate trade-offs or make conscious decisions about what should be done first based on business value and technical dependencies. Ahmed's team also struggled with legacy code maintenance: systems built years earlier by developers no longer with the company, with sparse documentation \textbf{(Transcript: 00:16:07--00:16:37)}. This hidden complexity was rarely accounted for when estimating work.

\section{Root Causes and Readiness for Change}
\label{sec:root-cause}

The fundamental problem was the absence of empirical process control. Djøf Trade Union made decisions based on assumptions rather than validated information. None of these assumptions were regularly tested against reality until it was too late. Teams didn't have regular checkpoints to demonstrate what had been built and gather feedback. There were no retrospectives where teams could reflect on what went wrong and agree on changes to prevent similar issues in the future.

% Figure: Wasted Effort Cycle - Vertical layout with side return loop
\begin{figure}[htbp]
\centering
\begin{tikzpicture}[
    step/.style={rectangle, rounded corners, draw, fill=blue!15, text width=4.5cm, align=center, minimum height=1cm, font=\small},
    problem/.style={rectangle, rounded corners, draw, fill=red!20, text width=4.5cm, align=center, minimum height=1cm, font=\small},
    arrow/.style={->, >=stealth, thick},
    label/.style={font=\scriptsize, text width=2.5cm, align=center}
]

% Vertical flow (top to bottom)
\node[step] (request) at (0,6) {Request arrives from business};
\node[step] (start) at (0,4) {Developer starts work (seems simple)};
\node[problem] (discover) at (0,2) {Hidden blocker discovered (week 2-3)};
\node[problem] (abandon) at (0,0) {Work abandoned (2-4 weeks wasted)};

% Arrows going down
\draw[arrow, blue] (request.south) -- (start.north) 
    node[midway, right, label, text=blue] {No refinement};
\draw[arrow, blue] (start.south) -- (discover.north) 
    node[midway, right, label, text=blue] {No feasibility check};
\draw[arrow, red, thick] (discover.south) -- (abandon.north) 
    node[midway, right, label, text=red] {Late discovery};

% Return loop on the left side
\draw[arrow, red, thick, dashed] (abandon.west) -- (-3,0) -- (-3,6) -- (request.west)
    node[midway, left, label, text=red] {No learning\\Pattern repeats};

% Problem examples on the right - moved further right with darker color
\node[font=\scriptsize, text=black!70, text width=3cm, align=left] at (5.5,3) {
    \textbf{Common blockers:}\\[0.5ex]
    • GDPR approval\\
    • Security access\\
    • Legacy issues\\
    • Vendor dependencies
};

\end{tikzpicture}
\caption{Wasted effort cycle before Scrum adoption. Work requests arrived without refinement or feasibility checks, developers started tasks that seemed simple, hidden blockers emerged late in development, and work was abandoned after 2-4 weeks of effort. Without retrospectives or process changes, this pattern repeated continuously.}
\label{fig:jawad-wasted-effort-cycle}
\end{figure}

Part of the root cause was also the absence of dedicated roles with clear accountability. No one owned the product backlog or facilitated the development process. They also lacked transparency into their capacity and work in progress. Without a visible backlog or sprint board, stakeholders couldn't see what the team was working on, creating frustration on both sides.

Finally, there was insufficient management support for changing how work was done. Research on Agile transformations demonstrates that management support is the second most critical success factor, creating conditions that enable developer effectiveness and continuous improvement to flourish \textit{\parencite{russo2021agile}}. Without executive buy-in and resources dedicated to process improvement, teams had little leverage to push for changes.

These root causes (lack of empirical process control, absence of defined roles, insufficient transparency, and limited management support) combined to create an environment where software development was chaotic and unpredictable. However, recognizing these problems created readiness for change. Failed projects like the membership checkbox task made the costs of their current approach visible to stakeholders. This awareness became the catalyst for exploring structured frameworks - precisely the context that led Djøf Trade Union to explore Scrum as a potential solution.

\chapter{Recognizing Development Challenges}
\label{ch:problem-identification}

Before Djøf Trade Union adopted Scrum, their development team faced challenges that hindered value delivery. Understanding these problems explains why they chose Agile practices. The Scrum framework emphasizes that transparency enables inspection and adaptation \textit{\parencite{schwaber2020scrum}}, but achieving transparency first requires identifying what is unclear. Empirical evidence demonstrates that teams struggle when organizational blockers around stakeholder alignment and continuous learning remain unaddressed \textit{\parencite{verwijs2021scrum}}.

\section{Key Challenges}
\label{sec:key-challenges}

Djøf Trade Union faced problems with how work arrived and was prioritized. Tasks emerged without consideration for capacity or strategic importance, creating an environment where developers constantly reacted rather than working proactively.

Legacy systems made this worse. Ahmed was asked to add a checkbox to the member registration system \textbf{(Transcript: 00:02:23)}. The legacy system had security restrictions preventing changes without logging in as an actual member, and creating test members was not permitted \textbf{(Transcript: 00:03:04)}. He spent weeks on this task, only to have it abandoned \textbf{(Transcript: 00:04:32)}.

The task needed legal approval and vendor coordination first \textbf{(Transcript: 00:04:48)}. No gatekeeping process ensured tasks were achievable before developers invested time. This pattern repeated: assignments looked simple but contained hidden dependencies or technical constraints.

% Figure: Problem Workflow - Before Scrum Adoption
% Simple narrative flow showing how work arrived chaotically

\begin{figure}[htbp]
    \centering
    \begin{tikzpicture}[
        box/.style={rectangle, draw, fill=blue!10, text width=2.8cm, align=center, minimum height=1cm, font=\small},
        problem/.style={rectangle, draw, fill=red!20, text width=2.8cm, align=center, minimum height=1cm, font=\small},
        arrow/.style={->, >=stealth, thick}
    ]

    % Top section - Business requests
    \node[box] (business1) at (0,4) {Business Request 1\\(GDPR unclear)};
    \node[box] (business2) at (3.5,4) {Business Request 2\\(No priority)};
    \node[box] (business3) at (7,4) {Business Request 3\\(Legacy constraints)};

    % Middle - Direct to developers (the problem)
    \node[problem] (devs) at (3.5,2) {Development Team\\(Overwhelmed)};

    % Arrows showing chaotic flow
    \draw[arrow, red] (business1) -- (devs);
    \draw[arrow, red] (business2) -- (devs);
    \draw[arrow, red] (business3) -- (devs);

    % Bottom - Outcomes
    \node[problem] (waste) at (1,0) {Wasted Effort\\(2-4 weeks lost)};
    \node[problem] (unclear) at (6,0) {Unclear Requirements\\(Late discovery)};

    \draw[arrow, red, dashed] (devs) -- (waste);
    \draw[arrow, red, dashed] (devs) -- (unclear);

    % Labels
    \node[above=0.1cm of business2, font=\small\bfseries] {Ad-hoc Work Arrival};
    \node[left=0.3cm of devs, font=\scriptsize, text=red, text width=1.8cm, align=right] {No filtering\\No priority\\No refinement};

    \end{tikzpicture}
    \caption{Chaotic work arrival before Scrum adoption. Business requests flowed directly to developers without proper refinement, prioritization, or feasibility analysis, leading to wasted effort and late discovery of blockers.}
    \label{fig:jawad-problem-workflow}
\end{figure}


\section{Problem Analysis}
\label{sec:problem-analysis}

The core issue was absence of structured processes connecting business needs with development capacity. No intermediary translated requests into feasible requirements. Business stakeholders assumed requests were feasible; developers accepted tasks without understanding context. Both operated with incomplete information.

Without a backlog or product owner, everything seemed equally urgent. No mechanism evaluated tradeoffs based on value and dependencies. Legacy code maintenance added complexity: systems built by departed developers with sparse documentation \textbf{(Transcript: 00:16:07)}.

\section{Root Causes and Readiness for Change}
\label{sec:root-cause}

The fundamental problem was absence of empirical process control. Decisions relied on assumptions rather than validated information. Teams had no checkpoints to demonstrate work or gather feedback. No retrospectives existed for reflection.

% Figure: Wasted Effort Cycle - Vertical layout with side return loop
\begin{figure}[htbp]
\centering
\begin{tikzpicture}[
    step/.style={rectangle, rounded corners, draw, fill=blue!15, text width=4.5cm, align=center, minimum height=1cm, font=\small},
    problem/.style={rectangle, rounded corners, draw, fill=red!20, text width=4.5cm, align=center, minimum height=1cm, font=\small},
    arrow/.style={->, >=stealth, thick},
    label/.style={font=\scriptsize, text width=2.5cm, align=center}
]

% Vertical flow (top to bottom)
\node[step] (request) at (0,6) {Request arrives from business};
\node[step] (start) at (0,4) {Developer starts work (seems simple)};
\node[problem] (discover) at (0,2) {Hidden blocker discovered (week 2-3)};
\node[problem] (abandon) at (0,0) {Work abandoned (2-4 weeks wasted)};

% Arrows going down
\draw[arrow, blue] (request.south) -- (start.north) 
    node[midway, right, label, text=blue] {No refinement};
\draw[arrow, blue] (start.south) -- (discover.north) 
    node[midway, right, label, text=blue] {No feasibility check};
\draw[arrow, red, thick] (discover.south) -- (abandon.north) 
    node[midway, right, label, text=red] {Late discovery};

% Return loop on the left side
\draw[arrow, red, thick, dashed] (abandon.west) -- (-3,0) -- (-3,6) -- (request.west)
    node[midway, left, label, text=red] {No learning\\Pattern repeats};

% Problem examples on the right - moved further right with darker color
\node[font=\scriptsize, text=black!70, text width=3cm, align=left] at (5.5,3) {
    \textbf{Common blockers:}\\[0.5ex]
    • GDPR approval\\
    • Security access\\
    • Legacy issues\\
    • Vendor dependencies
};

\end{tikzpicture}
\caption{Wasted effort cycle before Scrum adoption. Work requests arrived without refinement or feasibility checks, developers started tasks that seemed simple, hidden blockers emerged late in development, and work was abandoned after 2-4 weeks of effort. Without retrospectives or process changes, this pattern repeated continuously.}
\label{fig:jawad-wasted-effort-cycle}
\end{figure}

No one owned the product backlog or facilitated development. Without visible backlogs, stakeholders could not see team work, creating frustration.

Management support was insufficient. Research demonstrates that management support is the second most critical success factor for Agile transformations \textit{\parencite{russo2021agile}}. Without executive support, teams had little leverage for change.

These root causes created chaotic, unpredictable development. However, recognizing problems created readiness for change. Failed projects made costs visible to stakeholders, catalyzing exploration of Scrum.

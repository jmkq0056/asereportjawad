\chapter{Understanding Stakeholder Voices}
\label{ch:stakeholder-mapping}

Successful Agile implementation depends on understanding and engaging stakeholders effectively. Research with 2,000 Scrum teams found that stakeholder concern is a key driver of team effectiveness \textit{\parencite{verwijs2021scrum}}. Djøf Trade Union gradually learned to identify stakeholders and build collaborative relationships that support value delivery.

\section{Stakeholder Identification and Collaboration}
\label{sec:identifying-stakeholders}

The organization serves two groups: internal employees and union members. Internal employees use systems for booking management and account administration \textbf{(Transcript: 00:01:34--00:02:04)}. Union members interact with member-facing systems. These two stakeholder groups create challenges because internal and member needs don't always align.

The Product Owner became the bridge between stakeholder groups and the development team. Before Scrum, tasks arrived chaotically without clear understanding of who needed the work or why. The Product Owner now gathers needs from both groups, translates them into backlog items, and ensures the team understands context \textbf{(Transcript: 00:11:22--00:11:52)}.

They recognized that stakeholders have different influence and interest levels. Leadership cares about member growth and operational efficiency. They set budget priorities and decide which initiatives receive funding. Daily system users provide practical feedback about what works and what frustrates them.

% Figure: Stakeholder Landscape
% Simple narrative showing two stakeholder groups and Product Owner bridge

\begin{figure}[htbp]
    \centering
    \begin{tikzpicture}[
        stakeholder/.style={rectangle, draw, fill=blue!15, text width=2.5cm, align=center, minimum height=1cm, font=\small},
        po/.style={rectangle, draw, fill=green!25, text width=2.5cm, align=center, minimum height=1cm, font=\small\bfseries},
        team/.style={rectangle, draw, fill=orange!20, text width=2.5cm, align=center, minimum height=1cm, font=\small},
        arrow/.style={<->, >=stealth, thick}
    ]

    % Top section - Stakeholder groups
    \node[stakeholder] (internal) at (0,3.5) {Internal\\Employees};
    \node[stakeholder] (members) at (5,3.5) {Union\\Members};

    % Middle - Product Owner (the bridge)
    \node[po] (po) at (2.5,1.8) {Product Owner\\(Bridge)};

    % Bottom - Development Team
    \node[team] (team) at (2.5,0) {Development\\Team};

    % Arrows showing flow
    \draw[arrow, blue] (internal) -- (po);
    \draw[arrow, blue] (members) -- (po);
    \draw[arrow, green!60!black] (po) -- (team);

    % Labels
    \node[above=0.1cm of po, font=\scriptsize, text width=2cm, align=center] {Gathers needs\\Translates context};
    \node[right=0.3cm of team, font=\scriptsize, text width=2cm, align=left] {Receives\\prioritized\\backlog};

    \end{tikzpicture}
    \caption{Stakeholder landscape showing two groups served by Ahmed's organization. The Product Owner bridges stakeholder groups and the development team, gathering needs from both internal employees and union members, then translating them into prioritized backlog items.}
    \label{fig:jawad-stakeholder-landscape}
\end{figure}


Refinement created opportunities for stakeholders to shape solutions before development begins. The team now breaks down epics collaboratively with input from users \textbf{(Transcript: 00:05:30--00:06:00)}. When stakeholders participate in defining acceptance criteria, they gain realistic expectations.

Sprint reviews became regular touchpoints where stakeholders see completed work and provide feedback. The team invites business representatives to reviews, demonstrating features and gathering reactions \textbf{(Transcript: 00:20:37--00:21:07)}. Quarterly demos expanded this transparency by showcasing three months of progress to broader audiences.

Direct stakeholder-to-developer communication still happens. Ahmed adapted practically: when users approach him with ideas, he listens but redirects them to the Product Owner for prioritization \textbf{(Transcript: 00:29:22--00:30:21)}. This protects the team while recognizing valuable ideas.

\section{Managing Competing Interests and Communication}
\label{sec:competing-interests}

Stakeholders do not always agree. Economic constraints take priority: when leadership determines something costs too much, the project stops regardless of user enthusiasm \textbf{(Transcript: 00:36:17--00:36:47)}. Technical feasibility also matters. When requests would compromise system integrity, developers push back with Product Owner support.

The team learned to find middle ground. Understanding underlying needs often reveals alternative solutions. Successful Agile transformations require balancing multiple influences, with developer skills and management support being critical factors \textit{\parencite{russo2021agile}}.

Stakeholder management improved significantly with Agile practices. Structured refinement, regular reviews, transparent backlog, and clear Product Owner ownership created predictability that reduces frustration. Stakeholders know when to expect features. Management sees progress through transparent metrics rather than status reports that may hide problems.

The team established multiple feedback channels tailored to different stakeholder groups. Members receive newsletters announcing new features and changes, creating awareness without direct development team involvement. Internal stakeholders participate in sprint reviews and can observe the Jira backlog. Leadership receives quarterly progress updates aligned with strategic objectives. These layered communication approaches ensure each stakeholder group receives appropriate information without overwhelming the development team with coordination demands.

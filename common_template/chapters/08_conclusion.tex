\chapter{Conclusion}
\label{ch:conclusion}

Djøf Trade Union transformed chaotic development practices into structured, predictable delivery through pragmatic Scrum adoption. Their journey from ad-hoc task assignment to refined backlog management illustrates how Agile frameworks address fundamental organizational impediments when adapted thoughtfully to context.

The failed membership checkbox project crystallized the costs of absent process controls. Two weeks of development investment vanished because no gatekeeping process verified legal compliance or vendor capability before work began. This painful lesson catalyzed the adoption of refinement sessions that now surface blockers proactively. Their refinement practice demonstrates that breaking down epics collaboratively with clear acceptance criteria prevents wasted effort and aligns stakeholder expectations.

Their flexibility in adapting Scrum reveals maturity beyond rigid framework adherence. Dropping story points when stakeholders misunderstood them and occasionally skipping retrospectives when sprints lacked meaningful progress shows they value outcomes over ceremony. This pragmatism reflects understanding that frameworks serve teams rather than constrain them.

\section{Emerging Challenges}
\label{sec:emerging-challenges}

Legacy system complexity and test instability represent ongoing impediments. Flaky tests that fail randomly erode confidence and waste developer time. Limited monitoring visibility delays problem detection, particularly for overnight failures. These technical gaps suggest areas for continued investment.

\section{Future Development}
\label{sec:future-development}

The organization built strong foundations through refinement discipline, role clarity, and stakeholder collaboration. Enhancing observability, stabilizing test automation, and gradually reducing manual deployment gates will extend their Agile maturity. Their demonstrated willingness to adapt practices positions them well for continued evolution.
